%% This is emulateapj reformatting of the AASTEX sample document
%%
\documentclass[preprint2]{aastex6}

%\usepackage{graphicx}
%\usepackage[space]{grffile}
%\usepackage{latexsym}
%\usepackage{textcomp}
%\usepackage{longtable}
%\usepackage{multirow,booktabs}
%\usepackage{amsfonts,amsmath,amssymb}
%\usepackage{natbib}
% You can conditionalize code for latexml or normal latex using this.
%\newif\iflatexml\latexmlfalse
%\usepackage[utf8]{inputenc}
%\usepackage[ngerman,greek,english]{babel}

%\usepackage[english]{babel}
%\usepackage{natbib}

\newcommand{\e}{\mbox{e}^}
\newcommand{\amm}{NH$_3$}
\newcommand{\kms}{km\,s$^{-1}$}
\newcommand{\cc}{cm$^{-3}$}
\newcommand{\vlsr}{$v_{\mathrm{LSR}}$}
\newcommand{\sigv}{$\sigma_v$}
\newcommand{\tkin}{$T_\mathrm{kin}$}
\newcommand{\tex}{$T_\mathrm{ex}$}
\newcommand{\tmb}{$T_\mathrm{MB}$}
\newcommand{\nh}{N(H$_2$)}
\newcommand{\namm}{N(NH$_3$)}
\newcommand{\mvir}{$M_\mathrm{vir}$}
\newcommand{\avir}{$\alpha_\mathrm{vir}$}
%\newcommand{\micron}{$\mu$m}
%\newcommand{\arcmin}{$^\prime$}
%\newcommand{\arcsec}{$^{\prime\prime}$}

%\def\comment#1{\noindent{\bf \textcolor{red}{[#1]}}}


%% You can insert a short comment on the title page using the command below.
%\slugcomment{To appear in The Astrophysical Journal (ApJ)}

%% If you wish, you may supply running head information, although
%% this information may be modified by the editorial offices.
%% The left head contains a list of authors,
%% usually a maximum of three (otherwise use et al.).  The right
%% head is a modified title of up to roughly 44 characters.
%% Running heads will not print in the manuscript style.

\shorttitle{Transition to Coherence with GAS}
\shortauthors{J.E. Pineda et al.}

%% This is the end of the preamble.  Indicate the beginning of the
%% paper itself with \begin{document}.

\begin{document}

%% LaTeX will automatically break titles if they run longer than
%% one line. However, you may use \\ to force a line break if
%% you desire.

\title{Transition to Coherence in Dense Cores using 
the Green Bank Ammonia Survey (GAS)}


%% Use \author, \affil, and the \and command to format
%% author and affiliation information.
%% Note that \email has replaced the old \authoremail command
%% from AASTeX v4.0. You can use \email to mark an email address
%% anywhere in the paper, not just in the front matter.
%% As in the title, use \\ to force line breaks.

 \author{
Jaime E. Pineda\altaffilmark{1}, 
Rachel K. Friesen\altaffilmark{2}
Erik Rosolowsky\altaffilmark{3}, 
Felipe Alves\altaffilmark{1}, 
Ana Chac\'on-Tanarro\altaffilmark{1},
Hope How-Huan Chen\altaffilmark{4},
Michael Chun-Yuan Chen\altaffilmark{5},
James Di Francesco\altaffilmark{5,6},
Jared Keown\altaffilmark{5},
Helen Kirk\altaffilmark{6},
Anna Punanova\altaffilmark{1},
Young Min Seo\altaffilmark{7,8},
Yancy Shirley\altaffilmark{8},
Adam Ginsburg\altaffilmark{9},
Christine Hall\altaffilmark{10},
Stella S. R. Offner\altaffilmark{11},
Ayushi Singh\altaffilmark{12},
H\'ector G. Arce\altaffilmark{13},
Paola Caselli\altaffilmark{1},
Alyssa A. Goodman\altaffilmark{4},
%Fabian Heitsch\altaffilmark{14},
Peter G. Martin\altaffilmark{14},
Christopher Matzner\altaffilmark{12},
Philip C. Myers\altaffilmark{4},
Elena Redaelli\altaffilmark{1}
}

\altaffiltext{1}{Max-Planck-Institut f\"ur extraterrestrische Physik, Giessenbachstrasse 1, 85748 Garching, Germany}
\altaffiltext{2}{Dunlap Institute for Astronomy \& Astrophysics, University of Toronto, 50 St. George Street, Toronto, Ontario, Canada M5S 3H4} 
\altaffiltext{3}{Department of Physics, University of Alberta, Edmonton, AB, Canada} 
\altaffiltext{4}{Harvard-Smithsonian Center for Astrophysics, 60 Garden St., Cambridge, MA 02138, USA}
\altaffiltext{5}{Department of Physics and Astronomy, University of Victoria, 3800 Finnerty Road, Victoria, BC, Canada V8P 5C2}
\altaffiltext{6}{Herzberg Astronomy and Astrophysics, National Research Council of Canada, 5071 West Saanich Road, Victoria, BC, V9E 2E7, Canada}
\altaffiltext{7}{Jet Propulsion Laboratory, NASA, 4800 Oak Grove Dr, Pasadena, CA 91109, USA}
\altaffiltext{8}{Steward Observatory, 933 North Cherry Avenue, Tucson, AZ 85721, USA}
\altaffiltext{9}{National Radio Astronomy Observatory, Socorro, NM 87801, USA}
\altaffiltext{10}{Department of Physics, Engineering Physics \& Astronomy, Queen's University, Kingston, Ontario, Canada K7L 3N6}
\altaffiltext{11}{Department of Astronomy, University of Massachusetts, Amherst, MA 01003, USA}
\altaffiltext{12}{Department of Astronomy \& Astrophysics, University of Toronto, 50 St. George Street, Toronto, Ontario, Canada M5S 3H4}
\altaffiltext{13}{Department of Astronomy, Yale University, P.O. Box 208101, New Haven, CT 06520-8101, USA}
%\altaffiltext{14}{Department of Physics \& Astronomy, University of North Carolina at Chapel Hill, Chapel Hill, NC 27599-3255, USA}
\altaffiltext{14}{Canadian Institute for Theoretical Astrophysics, University of Toronto, 60 St. George St., Toronto, Ontario, Canada, M5S 3H8}



%% Notice that each of these authors has alternate affiliations, which
%% are identified by the \altaffilmark after each name.  Specify alternate
%% affiliation information with \altaffiltext, with one command per each
%% affiliation.

%\altaffiltext{1}{Visiting Astronomer, Cerro Tololo Inter-American Observatory.
%CTIO is operated by AURA, Inc.\ under contract to the National Science
%Foundation.}
%\altaffiltext{2}{Society of Fellows, Harvard University.}
%\altaffiltext{3}{present address: Center for Astrophysics,
%    60 Garden Street, Cambridge, MA 02138}
%\altaffiltext{4}{Visiting Programmer, Space Telescope Science Institute}
%\altaffiltext{5}{Patron, Alonso's Bar and Grill}

%% Mark off your abstract in the ``abstract'' environment. In the manuscript
%% style, abstract will output a Received/Accepted line after the
%% title and affiliation information. No date will appear since the author
%% does not have this information. The dates will be filled in by the
%% editorial office after submission.



%% Keywords should appear after the \end{abstract} command. The uncommented
%% example has been keyed in ApJ style. See the instructions to authors
%% for the journal to which you are submitting your paper to determine
%% what keyword punctuation is appropriate.

%% Authors who wish to have the most important objects in their paper
%% linked in the electronic edition to a data center may do so in the
%% subject header.  Objects should be in the appropriate "individual"
%% headers (e.g. quasars: individual, stars: individual, etc.) with the
%% additional provision that the total number of headers, including each
%% individual object, not exceed six.  The \objectname{} macro, and its
%% alias \object{}, is used to mark each object.  The macro takes the object
%% name as its primary argument.  This name will appear in the paper
%% and serve as the link's anchor in the electronic edition if the name
%% is recognized by the data centers.  The macro also takes an optional
%% argument in parentheses in cases where the data center identification
%% differs from what is to be printed in the paper.

%\keywords{globular clusters: general --- globular clusters: individual(\objectname{NGC 6397},
%\object{NGC 6624}, \objectname[M 15]{NGC 7078},
%\object[Cl 1938-341]{Terzan 8})}

%% From the front matter, we move on to the body of the paper.
%% In the first two sections, notice the use of the natbib \citep
%% and \citet commands to identify citations.  The citations are
%% tied to the reference list via symbolic KEYs. The KEY corresponds
%% to the KEY in the \bibitem in the reference list below. We have
%% chosen the first three characters of the first author's name plus
%% the last two numeral of the year of publication as our KEY for
%% each reference.


\begin{abstract}
We use the first data release (DR1) of the Green Bank Ammonia Survey (GAS). 
GAS is an ambitious Large Program at the Green Bank Telescope to map all regions within the northern hemisphere Gould Belt star-forming regions with $A_\mathrm{v} \gtrsim 7$ in emission from \amm\ and other key molecular tracers. 
This first release includes the data for four regions in Gould Belt clouds: L1688 in Ophiuchus, Orion A North in Orion, NGC 1333 in Perseus, and B18 in Taurus. 
We study the velocity dispersion obtained towards all these regions and identify sharp transitions between super-sonic and sub-sonic turbulence in molecular clouds.
\end{abstract}

\keywords{ISM: clouds --- stars: formation  --- ISM: molecules --- 
ISM: individual (Perseus Molecular Complex, L1451, HH211, IRAS03282)}

\section{Introduction}

\section{Results}


\section{Summary}
\label{sec:summary}


\acknowledgments
JEP, AP and PC acknowledge the financial support of the European Research Council (ERC; project PALs 320620). 
%
RKF is a Dunlap Fellow at the Dunlap Institute for Astronomy \& Astrophysics. The Dunlap Institute is funded through an endowment established by the David Dunlap family and the University of Toronto. 
%
EWR, PGM and CDM are supported by Discovery Grants from NSERC of Canada. 
%
SSRO acknowledges support from NSF grant AST-1510021. 
%
The National Radio Astronomy Observatory is a facility of the National Science Foundation operated under cooperative agreement by Associated Universities, Inc. 
%
%This research made use of \verb-Astropy-, a community-developed core Python package for Astronomy \citep{Robitaille_2013}, and \verb-matplotlib- \citep{Hunter_2007} software packages.

\facility{Green Bank Telescope} 

\software{Astropy \citep{Robitaille_2013}, Matplotlib \citep{Hunter_2007}, pyspeckit \citep{2011ascl.soft09001G}}

\bibliographystyle{apj}

\end{document}